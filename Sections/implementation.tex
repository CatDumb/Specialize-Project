%
% This is the TeX file for the implementation section
%
\section{IMPLEMENTATION}
\label{section:implementation}

As developed by the previous groups, the UTraffic architecture consists of four main components: an API server, a computing server, a database server, and an application interface. 

\begin{figure}[H]
    \centering
    \includegraphics[width=0.8\textwidth]{assets/images/Implementation/system_architecture_diagram.png}
    \caption{The system's architecture diagram.}
    \label{fig:utraffic_architecture}
\end{figure}

\begin{itemize}
    \item Database server: All the data of UTraffic is stored in a MongoDB database. The database server is responsible for storing and retrieving data. There is a lot of queries that concern geographical coordinates, and MongoDB well supports this type of data via the use of geospatial indexes.
    \item API server: The API server is a pivotal point of the UTraffic architecture, acting as a bridge to connect the application interface and the database server via API endpoints.
    \item Computing server: The computing server processes user data. This work consists of traffic status calculation, the user's credit, as well as the user's total traveled distance. We have not made any changes to this component in this phase of the project.
    \item Application: The application is the interface that the users interact with. Previous groups have developed the web application and the Android application. In this phase, we have developed a prototype iOS application.
\end{itemize}
Along with the main components, the Resource APIs component acts as a data mining source in the case of insufficient data. 

\subsection{Bus Services Integration}

\subsubsection{Getting a single route's information}

The integrity of the data on the Buyt TPHCM website is guaranteed due to its commissioning and maintenance by the Ho Chi Minh City Department of Transport. 

Several JavaScript source files power this site:
\begin{itemize}
    \item \textbf{L.Config.js:} This is a configuration file to set up the map rendering for Buyt TPHCM. The tiles are loaded from http://map.stis.vn/bright/{z}/{x}/{y}.png. This means that FPT has forked a copy of the OpenStreetMap project and is hosting it on their own server.
    \item \textbf{L.RouteMap.js:} This source file is responsible for showing the stops and bus routes. We were able to get the coordinate sets that we wanted by placing breakpoints on this file.
\end{itemize}

\begin{figure}[H]
    \centering
    \includegraphics[width=0.8\textwidth]{assets/images/Research/Bus/lat_long_web.png}
    \caption{Capturing and observing the route data using the DevTool.}
    \label{fig:lat_long_web}
\end{figure}

By inserting breakpoints in L.RouteMap.js, we can easily obtain the coordinates of the route, as demonstrated on the right-hand side. It is worth mentioning that while the coordinates are given in both distinct arrays and a single array with integrated data, it is more advantageous to utilize the latitudes and longitudes in separate arrays. The reason for this is that the present system implementation utilizes MongoDB, which does not allow tuples due to its implementation of BSON.

When it is desirable to work with a single array of coordinates, the stored data can be retrieved and combined together. Below is an example where we zip the arrays and plot them using Matplotlib (please note that since there are over 600 elements in each array, 668 to be exact, a figurative number was put in the code excerpt).

\begin{lstlisting}[basicstyle=\ttfamily\small, language=python]
    lat = [10.001, 10.002] # the latitudes
    lng = [106.001, 106.002] # the longitudes
    coordinates = [(lat[i], lng[i]) for i in range(min(len(lat), len(lng)))]
    # Extract latitudes and longitudes from the coordinates
    latitudes, longitudes = zip(*coordinates)
    
    # Create a scatter plot to mark the coordinates
    plt.scatter(longitudes, latitudes, color='red', marker='o', label='Coordinates')
    
    # Create a line plot to draw the polyline
    plt.plot(longitudes, latitudes, color='blue', label='Polyline')
    
    # Set labels for the x and y axes
    plt.xlabel('Longitude')
    plt.ylabel('Latitude')
    
    # Add a legend
    plt.legend()
    
    # Display the plot
    plt.show()
\end{lstlisting}

The result is as follow:
\begin{figure}[H]
    \centering
    \subfloat[The Matplotlib representation of the bus route 8.]{
        \includegraphics[width=0.4\linewidth]{assets/images/Research/Bus/plot_route.png}
        \label{fig:plot-route}
    }
    \hspace{0.5cm} % Add a horizontal space of 0.5 cm
    \subfloat[The actual route 8 from Buyt TPHCM.]{
        \includegraphics[width=0.4\linewidth]{assets/images/Research/Bus/actual_route.png}
        \label{fig:actual_route}
    }
\end{figure}

Similarly, information about the bus stops can be obtained by placing breakpoints in L.RouteMap.js. Each route comes with a list of stops and their corresponding data. This will be useful in designing our data schema later on when we integrate the gathered information into our database.

\begin{figure}[H]
    \centering
    \includegraphics[width=0.95\textwidth]{assets/images/Research/Bus/stop_web.png}
    \caption{Capturing and observing the bus stops data using the DevTool.}
    \label{fig:stop_data}
\end{figure}

\subsubsection{Automating the data collection process}

Since there are about 130 routes currently displayed on the website, it is impossible to place breakpoints and collect data by hand. This necessitates the use of automatic scripts to gather data.

After careful examinations, we found several places where we could automate getting the data. This introduces us to the routevar.js file, which contains the three functions: \lstinline{loadVarsByRou(rouId)}, \lstinline{loadStopsByVar(prtId, rouId, varId)}, and \lstinline{loadPathByVar(prtId, rouId, varId)}.

It is worth it for us to elaborate more on the given terms:
\begin{itemize}
    \item \lstinline{rouId}: This value represents a bus route's ID. It is important to note that this ID is not the same as the route number. For example, the route SWB1 is represented by the ID 337. 
    \item \lstinline{prtId}: This value represents a bus route's leg, which is either forward or return. The source from which we got the data does not use set values for all the routes, with some routes using 1 and 2 denoting each leg, while some other routes use 3, 4 or even 7, 8. 
    \item \lstinline{varId}: This value represents a bus route's variant. A variant's value is the combination of the \lstinline{rouId} and \lstinline{prtId}. Therefore, if the \lstinline{rouId} is 8 and the \lstinline{prtId} is 1, then the \lstinline{varId} is \textit{r8v1}.
\end{itemize}

The automated data retrieval can be described in the following steps:
\begin{enumerate}
    \item Go to the source website and wait for all routes to load.
    \item Capture all links to individual routes and store them in a temporary list.
    \item Extract the necessary ID in each link and loop over the list to extract individual data.
\end{enumerate}

Because the \lstinline{rouId} is present in a bus route view's URL, we can extract this value and log the value returned by the \lstinline{loadVarsByRou} function. After that, we can split the recorded string to get the \lstinline{prtId} value. The data retrieval process then involves injecting the website with two custom JavaScript variables that can capture the data loaded by the \lstinline{loadStopsByVar(prtId, rouId, varId)} and \lstinline{loadPathByVar(prtId, rouId, varId)}. This is done after we have injected the website with our custom \lstinline{routevar.js} script that assigns the source's data to the custom variables. This whole process is then automated using Selenium WebDriver. Figure \ref{fig:data_org} shows how the raw JSON data are stored after they are collected. This structured organization makes it easy for later scripts to iterate over the data and further process them.

\begin{figure}[H]
    \centering
    \includegraphics[width=0.9\textwidth]{assets/images/Implementation/data_org.png}
    \caption{Organization of the data retrieval process.}
    \label{fig:data_org}
\end{figure}

For the details of a bus route, this was made easy since they are explicitly displayed on the source website. We were able to parse the existing HTML structures and put the data into a JSON file and automate the process using Selenium WebDriver.

\subsubsection{Data storage of bus services}
Given the structured nature of the data we collected, as well as the existing use of MongoDB, it was a straightforward process to load the collection JSON files into MongoDB. This process was made even simpler with MongoDB Compass, a GUI for MongoDB that allows us to import JSON and CSV files into the database with a few clicks.

\begin{figure}
    \centering
    \includegraphics[width=0.75\textwidth]{assets/images/Implementation/raw_stops.png}
    \caption{The bus stops of a route before processing.}
    \label{fig:raw_stops}
\end{figure}

However, the first thing to do was to preprocess the data for efficient and logical storage. In particular, for a route number, the raw data has the stops listed in full details, as in Figure \ref{fig:raw_stops}. If we had gone and stored the data as-is to MongoDB, one clear problem would have been duplicated data. Therefore, our solution to this was to create a Python script to iterate over the raw JSON and pick out the unique stops and save them to a CSV file. For the association with the routes, we reduced the stops array to only contain the IDs. This also helped save the order of the stops along a certain path. Finally, when it comes to querying, the IDs can be used to retrieve the full details of the stops from the stops collection, improving performance.

We also implemented a \lstinline{2dsphere} index on the locations of the bus stops. This helps tremendously in the querying processes that care about the radius from within a coordinate. Our \textit{Get nearby bus stops} API endpoint is a prime example of how useful this index is, because if it was not implemented, the scanning process would have been very slow and inefficient. A column scan would have been used, and given that there are over 4000 stops in the database, the performance would have been unacceptable. The index creation was straightforward with the help of MongoDB Compass.

We decided to have 4 new collections within the existing MongoDB database. They are as follows:
\begin{itemize}
    \item \textbf{Bus\_Associations}: This collection associates bus routes with their stops. The stops are stored as an array of IDs, preserving the order of the stops along a certain route.
    \item \textbf{Bus\_Metadata}: This collection contains all information about bus routes, including their names, operating hours, operating agencies, the daily total trips, and the trip spacings.
    \item \textbf{Bus\_Path\_Coordinates}: This collection stores the paths of the bus routes in 4 arrays of forward and return latitudes and longitudes.
    \item \textbf{Bus\_Stops}: This collection stores all the bus stops in the database. The details of a stop are its official street address, its location, its type, whether it is disability-friendly, and the routes that pass through it.
\end{itemize}

Figure \ref{fig:dbcompass} showcases the Bus\_Stops collection in MongoDB Compass with 4414 documents.

\begin{figure}[H]
    \centering
    \includegraphics[width=0.95\textwidth]{assets/images/Implementation/db_compass.png}
    \caption{The Bus\_Stops collection in MongoDB Compass.}
    \label{fig:dbcompass}
\end{figure}


\subsection{Bus API Development and Server Deployment}
We have previously discussed the introduction of the API additions in the UTraffic system, and we have taken the steps to deploy them on the live server of UTraffic. The endpoints are now live and working as intended.
\subsubsection{The Bus API}
The following are the details of the new endpoints, an example URL and the corresponding response.

\begin{itemize}
    \item \textbf{Get nearby bus stops}.
    \begin{itemize}
        \item URL: \url{https://api.bktraffic.com/api/bus/nearby-stops}.
        \item Method: GET.
        \item Query parameters: \lstinline{lat}, \lstinline{lng}, \lstinline{radius}.
        \item Example request URL: \url{https://api.bktraffic.com/api/bus/nearby-stops?lat=10.7721&lng=106.6579&radius=500}
        \item Example response: 
        \begin{figure}[H]
            \centering
            \includegraphics[width=0.4\linewidth]{assets/images/Implementation/nearby_bus_response.png}
            \caption{The response containing bus stops within 500m of the coordinate (10.7721, 106.6579).}
            \label{fig:nearby_stop_response}
        \end{figure}
    \end{itemize}

    \item \textbf{Get the details of a bus stop}.
    \begin{itemize}
        \item URL: \url{https://api.bktraffic.com/api/bus/stop-details}.
        \item Method: GET.
        \item Query parameters: \lstinline{id}.
        \item Example request URL: \url{https://api.bktraffic.com/api/bus/stop-details?id=2931}.
        \item Example response:
        \begin{figure}[H]
            \centering
            \includegraphics[width=0.38\textwidth]{assets/images/Implementation/stop_detail_response.png}
            \caption{The response containing the details of the bus stop with the ID 2931.}
            \label{fig:stop_detail_response}
        \end{figure}
    \end{itemize}

    \item \textbf{Get nearby disability friendly bus stops}.
    \begin{itemize}
        \item URL: \url{https://api.bktraffic.com/api/bus/disability-friendly-stops}.
        \item Method: GET.
        \item Query parameters: \lstinline{lat}, \lstinline{lng}.
        \item Example request URL: \url{http://api.bktraffic.com/api/bus/disability-friendly-stops?lat=10.77037&lng=106.69868}
        \item Example response:
        \begin{figure}[H]
            \centering
            \includegraphics[width=0.5\textwidth]{assets/images/Implementation/nearby_disable_response.png}
            \caption{The response containing bus stops that are disability-friendly near the coordinate (10.77037, 106.69868).}
            \label{fig:nearby_disable_response}
        \end{figure}
    \end{itemize}

    \newpage
    \item \textbf{Get all bus routes}.
    \begin{itemize}
        \item URL: \url{https://api.bktraffic.com/api/bus/routes}.
        \item Method: GET.
        \item Query paramaters: none.
        \item Example request URL: \url{https://api.bktraffic.com/api/bus/routes}.
        \item Example response:
        \begin{figure}[H]
            \centering
            \includegraphics[width=0.4\textwidth]{assets/images/Implementation/all_routes_response.png}
            \caption{The response containing all the bus routes of Ho Chi Minh City.}
            \label{fig:all_routes_response}
        \end{figure}
    \end{itemize}

    \item \textbf{Get the bus stops along a bus route's path}.
    \begin{itemize}
        \item URL: \url{https://api.bktraffic.com/api/bus/route-stops}.
        \item Method: GET.
        \item Query parameters: \lstinline{id}, \lstinline{leg}.
        \item Example request URL: \url{https://api.bktraffic.com/api/bus/route-stops?id=08&leg=forward}.
        \item Example response:
        \begin{figure}[H]
            \centering
            \includegraphics[width=0.4\textwidth]{assets/images/Implementation/route_stops_response.png}
            \caption{The response containing the array of the bus stop IDs along the route 08.}
            \label{fig:route_stops_response}
        \end{figure}
    \end{itemize}

    \item \textbf{Get the details of a bus route}.
    \begin{itemize}
        \item URL: \url{https://api.bktraffic.com/api/bus/route-details}.
        \item Method: GET.
        \item Query parameters: \lstinline{id}.
        \item Example request URL: \url{http://api.bktraffic.com/api/bus/route-info?id=08}
        \item Example response:
        \begin{figure}[H]
            \centering
            \includegraphics[width=0.45\textwidth]{assets/images/Implementation/route_details_response.png}
            \caption{The response containing the details of the bus route with the ID 08.}
            \label{fig:route_details_response}
        \end{figure}
    \end{itemize}

    \item \textbf{Get the path of a bus route}.
    \begin{itemize}
        \item URL: \url{https://api.bktraffic.com/api/bus/route-path}.
        \item Method: GET.
        \item Query paramaters: \lstinline{id}, \lstinline{leg}.
        \item Example request URL: \url{https://api.bktraffic.com/api/bus/route-path?id=08&leg=forward}.
        \item Example response:
        \begin{figure}[H]
            \centering
            \includegraphics[width=0.45\textwidth]{assets/images/Implementation/route_path_response.png}
            \caption{The response containing the line string path of the bus route 08.}
            \label{fig:route_path_response}
        \end{figure}
    \end{itemize} 
\end{itemize}

\subsubsection{Deployment of the Bus API}
The deployment process was made possible first by accessing the server via the SSH protocol. The server is running Ubuntu 20.04 LTS with the NodeJS version 12.22.12 installed. It was crucial to first upgrade this NodeJS version to a more recent one due to this version being out of the active support. We have upgraded the NodeJS version to version 16.16.0 (LTS) instead of the newer version 18 or above, because the server hardware is also not very powerful, and the newer versions of NodeJS are not compatible. We also noticed that there was no Node Version Manager (NVM) installed on the server. This is a tool that allows us to install and manage multiple versions of NodeJS on the same machine, therefore, we installed the new node version with NVM.

The next step was to upgrade PM2. PM2 is a process manager for NodeJS applications. It allows us to keep the application running in the background, restart the application automatically when it crashes, and monitor the application's resource usage. We have upgraded PM2 to accompany the new NodeJS version successfully.

We use GitLab to host our source code. After the code was ready on the GitLab repository, we issued a \lstinline{git pull} on the server's terminal to pull the latest code to the correct directory. We then installed the dependencies using \lstinline{npm install} and restarted the process with \lstinline{pm2 restart BKTrafficApi}. Figure \ref{fig:pm2_show} shows the running server process.

\begin{figure}[H]
    \centering
    \includegraphics[width=0.8\textwidth]{assets/images/Implementation/pm2_show.png}
    \caption{The server information and the running NodeJS process.}
    \label{fig:pm2_show}
\end{figure}

\subsection{Concepts for parking services}

In addition to advancing bus services, our group aims to implement functionalities related to parking. Due to the absence of a centralized data source, a model similar to the one adopted by Viettel (referenced in \ref{section:related_research}) is deemed more suitable for populating our database with information on available parking lots. However, to circumvent financial complexities, incentives offered to parking lot owners will be confined to analytics and the capability to promote their parking spaces to the application's user base.

\subsubsection{Functional requirements}

\begin{enumerate}
    \item \textbf{User Registration and Authentication}:
    \begin{itemize}
        \item Users (parking lot owners) must be able to register and create accounts securely.
        \item Authentication mechanisms, such as email verification or two-factor authentication, should be in place.
    \end{itemize}
    \item \textbf{Space Listing}: Parking lot owners should be able to list their available parking spaces, specifying details such as location, pricing, availability, and any special features.
    \item \textbf{Photo Upload}: Users should be able to upload images of their parking spaces to provide visual information to potential customers.
    \item \textbf{Search and Filterting}:
    \begin{itemize}
        \item Users should be able to search for parking spaces based on location, availability,
        pricing, and other criteria.
        \item Filtering options should be provided to refine search results.
    \end{itemize}
    \item \textbf{Real-Time Availability Information}: The module should display real-time availability status of parking spaces, indicating whether a space is currently vacant or booked.
\end{enumerate}

\subsubsection{Non-functional requirements}

\begin{enumerate}
    \item \textbf{Performance}: The module should be responsive and perform efficiently, even during peak usage times
    \item \textbf{Security}:
        \begin{itemize}
            \item Ensure the security of user data, payment information, and sensitive details.
            \item Implement secure authentication and authorization mechanisms.
        \end{itemize}
    \item \textbf{Scalability}: The system should be able to scale with an increasing number of users, parking spaces, and transactions.
    \item \textbf{Usability}: The user interface should be intuitive and user-friendly for both parking lot owners and renters.
    \item \textbf{Reliability}: Ensure that the system is available and reliable, with minimal downtime for maintenance.
    \item \textbf{Cross-Platform Compatibility}: The module should be accessible on various devices and platforms, including web browsers and mobile applications.
    \item \textbf{Geolocation Accuracy}: Ensure that the geolocation feature accurately identifies the
    location of parking spaces on the map.
    \item \textbf{Error Handling}: Implement robust error handling to provide clear and informative error messages
\end{enumerate}

\subsubsection{Use cases}

\begin{enumerate}
    \item \textbf{List parking spaces}
    \begin{itemize}
        \item \textbf{Actor}: Parking lot owner
        \item \textbf{Description}: Parking lot owners can list their available parking spaces by providing details such as location, pricing, availability, and any special features. They can also upload photos of the space.
    \end{itemize}
    \item \textbf{Manage parking listings}
    \begin{itemize}
        \item \textbf{Actor}: Parking lot owner
        \item \textbf{Description}: The parking lot owner can view, edit, update, or remove their parking space listings. They can adjust pricing, mark spaces as unavailable, or delete listings as needed.
    \end{itemize}
    \item \textbf{Search for parking spaces}
    \begin{itemize}
        \item \textbf{Actor}: Customer
        \item \textbf{Description}: Users looking for parking spaces can search for available options based on location, date, time, and price. They can also use filters to narrow down the search results.
    \end{itemize}
    \item \textbf{View parking space details}
    \begin{itemize}
        \item \textbf{Actor}: Parking lot owner, customers
        \item \textbf{Description}: Both parking lot owners and users can see the real-time availability status of parking spaces
    \end{itemize}
    \item \textbf{Access account settings}
    \begin{itemize}
        \item \textbf{Actor}: Parking lot owner, customers
        \item \textbf{Description}: Users can access their account settings to view and edit their profile information and change their password.
    \end{itemize}
\end{enumerate}

Below is a diagram recapping these usecases

\begin{figure}[H]
    \centering
    \includegraphics[width=0.7\linewidth]{assets/images/Implementation/Parking/Parking Use case updated.drawio.png}
    \caption{Activity diagram of the module.}
    \label{fig:parking_usecase}
\end{figure}

\subsubsection{Diagrams}

\subsubsubsection{Sequence diagrams}

\begin{figure}[H]
    \centering
    \includegraphics[width=\linewidth]{assets/images/Implementation/Parking/Search_lot_sequence.png}
    \caption{Sequence diagram for searching nearby parking lots.}
    \label{fig:parking_search}
\end{figure}

\begin{figure}[H]
    \centering
    \includegraphics[width=\linewidth]{assets/images/Implementation/Parking/Availability.drawio.png}
    \caption{Sequence diagram for checking parking lot availability.}
    \label{fig:parking_availability}
\end{figure}

The ability for users to promote their own parking spaces for rental is akin to the previous features, with an additional verification process to ensure essential information is provided

\begin{figure}[H]
    \centering
    \includegraphics[width=0.75\linewidth]{assets/images/Implementation/Parking/List and Manage.drawio.png}
    \caption{Sequence diagram for advertising parking spaces.}
    \label{fig:parking_search}
\end{figure}

\subsubsubsection{Activity diagram}

For the act of searching and observing various parking spots, activity diagrams are unnecessary because these tasks involve querying without decision-making. Activity diagrams are provided for adding new parking lots and modifying the number of available parking spots

\begin{figure}[H]
    \centering
    \subfloat[Acitivity diagram for adding new parking lots.]{
        \includegraphics[width=0.3\textwidth]{assets/images/Implementation/Parking/Add parking activity.drawio.png}
        \label{fig:parking_add}
    }
    \hspace{1cm}
    \subfloat[Adjusting the number of available spots.]{
        \includegraphics[width=0.45\textwidth]{assets/images/Implementation/Parking/Adjusting available activity.drawio.png}
        \label{fig:parking_adjust}
    }
\end{figure}

\subsubsubsection{ERD diagram}

In the realm of parking lot management, a well-structured Entity-Relationship Diagram (ERD) plays a crucial role in database system development. This ERD is thoughtfully crafted to address the specific requirements of parking lot management, offering a clear depiction of the connections between various entities. Central to this ERD is the ”owns” relationship, serving as a linchpin to associate weak entities, representing parking lots, with their respective owners. These parking lots exhibit a diverse range of attributes, including name, address, geographical coordinates, the count of available parking spaces, and even the capacity to store visual references through photos. Although the server do have other information for our users, this diagram only includes their UID to serve as a primary key.

\begin{figure}[H]
    \centering
    \includegraphics[width=0.8\linewidth]{assets/images/Implementation/Parking/Parking ERD.drawio.png}
    \caption{The ERD for storing parking lots and their relationships.}
    \label{fig:parking_erd}
\end{figure}

\subsection{iOS App Development}
The iOS app is developed using Swift 5.9 and Xcode 15. The app is compatible with iOS 16.0 and above.

First and foremost, this app must at least offer the basic functionalities like that of the Android counterpart. This includes authentication, viewing the traffic status report, and the ability to communicate with the server. Other requirements are that the app must also be lightweight, performant, and consistent with the ongoing design of the Android app to an extent. During the development of this prototype, we have also taken into consideration the possibility of future expansion, such as adding more features and improving the user experience, as well as following Apple's best practices when it comes to iOS app development.

However, the similarities between the two versions should only be implemented to an extent, given the core differences between the two operating systems. We do not expect the complete iOS app to be \textit{identical} to its Android counterpart, nor the Android app's new functionalities such as the bus services to be the same as in iOS. 


\subsubsection{The landing page and permission requests}
Because the UTraffic app must know the user's real-time location in order to gather GPS data and also display nearby traffic status, it is crucial that we ask the users for location permissions upon the first app launch. This is done by presenting a dialog box that asks for the user's permission to access their location. The choice is saved into \lstinline{UserDefaults}. \lstinline{UserDefaults} is an interface to the user's defaults database, where key-value pairs of data are stored persistently across launches of the app. \cite{a2019_userdefaults}. Therefore, this process will not be repeated on subsequent app launches. The onboarding process is divided into three main phases: greetings, location request, and always-on location request. They are as follows (the texts are in Vietnamese to first accommodate the Android app's design, but we are adding an option for language change in the next phase):

\begin{figure}[H]
    \centering
    \subfloat[Greetings.]{
        \includegraphics[width=0.25\linewidth]{assets/images/Implementation/onboard1.png}
        \label{fig:onboard1}
    }
    \subfloat[Asking for location permission.]{
        \includegraphics[width=0.25\linewidth]{assets/images/Implementation/onboard2.png}
        \label{fig:onboard2}
    }
    \subfloat[Asking for always-on location permission.]{
        \includegraphics[width=0.25\linewidth]{assets/images/Implementation/onboard3.png}
        \label{fig:onboard3}
    }
    \caption{The onboarding process of the iOS UTraffic app.}
\end{figure}

\subsubsection{Account registration and login}
We adapt most of the user interface from the Android app to the iOS app in order to provide a consistent experience of the app across platforms. The account registration and login screens are as follows:
\begin{figure}[H]
    \centering
    \subfloat[The iOS account registration view.]{
        \includegraphics[width=0.25\linewidth]{assets/images/Implementation/register_view.png}
        \label{fig:ios_register_view}
    }
    \hspace{2cm}
    \subfloat[The iOS app's login view.]{
        \includegraphics[width=0.25\linewidth]{assets/images/Implementation/login_view.png}
        \label{fig:ios_login_view}
    }
    \label{fig:ios_auth_views}
    \caption{The authentication views on the iOS UTraffic app.}
\end{figure}


\subsubsection{Viewing the traffic status report}
We adopted the display logic from the Android app, where the northeast and southwest bounds of the map's visible area, along with the street level, are taken into account. However, since the Android app uses the original response data provided by the \lstinline{/get-status-v2} endpoint, using it on the iOS app means we have to develop additional classes to parse the data into the format that the iOS app can understand. We implemented the \lstinline{/get-status-v3} for this exact reason to support the iOS app better.

\begin{figure}[H]
    \centering
    \includegraphics[width=0.28\textwidth]{assets/images/Implementation/traffic_status_view.png}
    \caption{The home screen with the traffic status displayed.}
    \label{fig:traffic_status_view}
\end{figure}

\subsubsection{Basic bus browsing}
In this phase, we have implemented the bus list view within our iOS app. It is a view that displays all the routes within the database via the \lstinline{/routes} endpoint. 
\begin{figure}[H]
    \centering
    \includegraphics[width=0.28\textwidth]{assets/images/Implementation/bus_view.png}
    \caption{The bus list view displayed in a bottom sheet.}
    \label{fig:bus_view_ios}
\end{figure}


\subsubsection{Integration with HCMUT TrafficView}
We have developed an UI to see how the elements of HCMUT TrafficView would look like on iOS. The result is as follows:

\begin{figure}[H]
    \centering
    \includegraphics[width=0.3\textwidth]{assets/images/Implementation/hcmut_trafficview.png}
    \caption{The traffic camera view within the iOS app.}
    \label{fig:trafficview_ios}
\end{figure}

Because of mobile devices' limited processing power, it is best that we only show a snapshot of the camera at the time the user issues the request. This is to avoid programming complexity of having to deal with stream data, as well as to avoid the app from crashing due to memory issues.