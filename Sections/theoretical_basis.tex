\section{THEORETICAL BASIS}

\subsection{Model}

\subsubsection{Model-View-Controller (MVC)}

MVC is an architectural design pattern for building applications that separates the application into three main logical components: the model, the view, and the controller. In the MVC model:
\begin{itemize}
    \item \textbf{Model}: Represents the data and business logic of the application. It is the "unchanging essence" of the application and should be stable and long-lived.

    \item \textbf{View}: Represents the user interface of the application. There can be multiple views for different devices, such as a web interface, mobile app, or command line interface.

    \item \textbf{Controller}: Handles user input and updates the model and view accordingly. It acts as an intermediary between the model and the view, processing business logic and managing data flow.
\end{itemize}

\begin{figure}[H]
    \centering
    \includegraphics[width=0.8\textwidth]{assets/images/Research/System/MVC.png}
    \caption{MVC architecture}
    \label{fig:mvc}
\end{figure}

Benefits of using MVC:

\begin{itemize}
    \item \textbf{Separation of concerns}: Each component has a well-defined role, making the code easier to understand and maintain.
    \item \textbf{Improved testability}: Each component can be tested independently.
    \item \textbf{Increased flexibility}: Changes to one component can be made without affecting the others.
    \item \textbf{Easier collaboration}: Different developers can work on different components independently.
\end{itemize}

\subsubsection{Model-View-ViewModel (MVVM)}



\subsection{Frontend}

\subsubsection{Java}



\subsubsection{Swift}



\subsubsection{SwiftUI}



\subsection{Backend}

\subsubsection{Java}



\subsubsection{Python}



\subsection{Database}

\subsubsection{MySQL}



\subsubsection{MongoDB}



\subsubsection{Oracle}



\subsubsection{PostgreSQL}

