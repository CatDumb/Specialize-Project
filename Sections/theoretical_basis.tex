\section{THEORETICAL BASIS}

\subsection{OpenStreetMap's data format}
% Khoa viet cai nay nhe, noi ve trunk, tag, way, node... nhu trong bao cao cuoi ki cua may anh chi nhe


\subsection{Model}

\subsubsection{Model-View-Controller (MVC)}

MVC is an architectural design pattern for building applications that separates the application into three main logical components: the model, the view, and the controller. In the MVC model:

\begin{itemize}
    \item \textbf{Model}: Represents the data and business logic of the application. It is the "unchanging essence" of the application and should be stable and long-lived.

    \item \textbf{View}: Represents the user interface of the application. There can be multiple views for different devices, such as a web interface, mobile app, or command line interface.

    \item \textbf{Controller}: Handles user input and updates the model and view accordingly. It acts as an intermediary between the model and the view, processing business logic and managing data flow.
\end{itemize}

\begin{figure}[H]
    \centering
    \includegraphics[width=\textwidth]{assets/images/Research/System/MVC.png}
    \caption{MVC architecture}
    \label{fig:mvc}
\end{figure}

\subsubsection{Model-View-ViewModel (MVVM)}

MVVM architecture offers two-way data binding between view and view-model. The view-model makes use of observer pattern to make changes in the view-model. The components of MVVM are:

\begin{itemize}
    \item \textbf{Model}: Holds the data and business logic. The model is independant of UI or user interaction.
    \item \textbf{View}: Represents how the user interacts with the application.
    \item \textbf{View-model}: Acts as a bridge between the view and the model. It provides data from the model to the view and handles user interaction with the view.
\end{itemize}

\begin{figure}[H]
    \includegraphics[width=\textwidth]{assets/images/Research/System/MVVM.png}
    \caption{MVVM architecture}
    \label{fig:mvvm}
\end{figure}

\subsubsection{Conclusion}


\begin{table}[H]
    \centering
    \begin{tabular}{| c | p{0.4\textwidth} | p{0.4\textwidth} |}
        \hline
        \multicolumn{1}{|c|}{}
        & \multicolumn{1}{c|}{Advantage}
        & \multicolumn{1}{c|}{Disadvantage} \\ \hline
        \multirow{7}{*}{MVC}     
                &   \begin{itemize}[leftmargin=*,topsep=0pt,partopsep=0pt,parsep=0pt]
                        \item The development of the various components can be performed parallelly.
                        \item It avoids complexity by dividing an application into separate (MVC) units
                        \item It only uses a front controller pattern that processes web application requests using a single controller.
                        \item It provides a clean separation of concerns.
                    \end{itemize}
                &   \begin{itemize}[leftmargin=*,topsep=0pt,partopsep=0pt,parsep=0pt]
                        \item Business logic is mixed with Ul
                        \item Hard to reuse and implement tests
                        \item There is a need for multiple programmers to conduct parallel programming.
                    \end{itemize} \\ \hline
        \multirow{5}{*}{MVVC}
                &   \begin{itemize}[leftmargin=*,topsep=0pt,partopsep=0pt,parsep=0pt]
                        \item Business logic is decoupled from Ul.
                        \item Easy to reuse components.
                        \item You can write unit test cases for both the viewmodel and Model layer without the need to reference the View.
                    \end{itemize}
                &   \begin{itemize}[leftmargin=*,topsep=0pt,partopsep=0pt,parsep=0pt]
                        \item Maintenance of lots of codes in controller
                        \item Some people think that for simple UIs of MVVM architecture can be overkill
                    \end{itemize} \\ \hline
    \end{tabular}
    \caption{Comparison between MVC and MVVM}
\end{table}

Base on the comparison above, MVVM is more suitable for this project because it is easier to reuse components and write unit test cases.

\subsection{Frontend}

\subsubsection{Objective-C}

Objective-C currently ranks 22nd in popularity according to the TIOBE Index. While its use is declining, it remains significant in the Apple ecosystem, with around 60,000 active repositories. Introduced in the 1980s by Brad Cox and Tom Love, Objective-C is an object-oriented programming language primarily used for Apple platforms. It was licensed by NeXT Computer Inc., a company founded by Steve Jobs. Noteworthy features include classes, inheritance, dynamic binding, and a Smalltalk-like syntax for messaging objects. Despite declining usage, Objective-C remains one of the more popular programming languages in the market

Some of the great features of this language are:

\begin{itemize}
    \item \textbf{Object-Oriented}: Supports classes, objects, and object-oriented concepts.
    \item \textbf{Dynamic Binding}: Resolves method calls at runtime for flexibility.
    \item \textbf{Message Passing}: Utilizes message passing for a natural coding style.
    \item \textbf{Smalltalk-style Syntax}: Simple and expressive syntax like Smalltalk for clear code.
    \item \textbf{Cross-Platform}: Works on macOS, iOS, and other Apple platforms, plus Windows and Linux.
    \item \textbf{Interoperability with C}: Integrates seamlessly with C libraries for collaboration.
\end{itemize}

\subsubsection{Swift}

Swift is a powerful and modern programming language developed by Apple for building applications across its ecosystem, including iOS, macOS, watchOS, and tvOS. Introduced in 2014, Swift was designed to be efficient, expressive, and easy to learn, addressing the shortcomings of its predecessor, Objective-C. With a syntax that is both concise and readable, Swift empowers developers to create robust and high-performance applications. Known for its safety features, dynamic capabilities, and versatility, Swift has quickly become the language of choice for many developers seeking to craft innovative and seamless experiences within the Apple ecosystem. In this introduction, we'll explore the key features and characteristics that make Swift a standout language in the world of software development.

\subsubsection{SwiftUI}

In summary, SwiftUI, developed by Apple, has become increasingly popular among iOS app developers for valid reasons. Introduced five years after Swift alongside Swift 5 and Xcode 11, it serves as a UI toolkit designed for creating software across various platforms, including iOS, macOS, watchOS, and tvOS. SwiftUI offers a declarative approach to UI design, enabling developers to describe layout and behavior using a simple and intuitive syntax. This results in more efficient and faster development of complex UIs, thanks to concise and easily readable code compared to traditional imperative approaches.

In summary, SwiftUI stands out with these key features:

\begin{itemize}
    \item \textbf{Declarative Syntax}: Describes UI appearance easily, improving code readability.
    \item \textbf{Automatic Layout}: Handles UI layout automatically, reducing manual effort.
    \item \textbf{Dynamic UI}: Enables the creation of dynamic and interactive interfaces.
    \item \textbf{Cross-platform Support}: Works across all Apple devices with a single codebase.
    \item \textbf{Real-time Preview}: Provides live previews of UI changes during development.
    \item \textbf{Pre-built Components}: Offers ready-to-use UI components for quick and efficient development.
\end{itemize}

\subsubsection{Conclusion}

\begin{table}[H]
    \centering
    \begin{tabular}{| >{\centering\arraybackslash}m{2cm} | >{\centering\arraybackslash}m{6cm} | >{\centering\arraybackslash}m{6cm} |}
        \hline
        \textbf{Feature} & \textbf{Objective C} & \textbf{Swift} \\ \hline
        Age & Developed in the early 1980s & Introduced in 2014 \\ \hline
        Syntax & Uses C-based syntax with Smalltalk-style messaging & Uses a modern and concise syntax \\ \hline
        Performance & Slower than Swift due to overhead and lack of optimization & Faster than Objective-C due to optimization features \\ \hline
        Memory Managment & Prone to memory leaks & No memory leaks due to being type-safe and memory-safe \\ \hline
        Stability & Stable & Unstable as it is still growing \\ \hline
    \end{tabular}
    \caption{Comparison between Objective C and Swift}
\end{table}

Base on the comparison above, Swift is more suitable for this project because it is superior in multiple aspects when comparing to Objective C.

\subsection{Backend}

\subsubsection{Java}



\subsubsection{Python}


\subsubsection{NodeJS}

\subsubsection{Conclusion}

\subsection{Database}

\subsubsection{MySQL}

MySQL is a widely used database for web-based applications, offered as freeware with regular upgrades to improve capabilities and security. The freeware edition emphasizes efficiency and dependability over a comprehensive range of functionalities. Prominent characteristics encompass the ability to select storage engines, a user-friendly interface, and effective handling of extensive datasets in batches, all while maintaining modest resource use.

\begin{table}[H]
    \centering
    \begin{tabular}{| c | p{0.4\textwidth} | p{0.4\textwidth} |}
        \hline
        \multicolumn{1}{|c|}{}
        & \multicolumn{1}{c|}{Advantage}
        & \multicolumn{1}{c|}{Disadvantage} \\ \hline
        \multirow{7}{*}{,MySQL}     
                &   \begin{itemize}[leftmargin=*,topsep=0pt,partopsep=0pt,parsep=0pt]
                        \item MySQL is accessible for free, making it cost-effective for various applications.
                        \item Multiple user interfaces can be implemented, enhancing usability.
                        \item Supports both structured data (SQL) and semi-structured data (JSON).
                    \end{itemize}
                &   \begin{itemize}[leftmargin=*,topsep=0pt,partopsep=0pt,parsep=0pt]
                        \item The process of setting up MySQL for specific activities may necessitate a greater investment of time and effort in comparison to systems that automate certain functions, such as incremental backups.
                        \item While support is available for the free version, premium support may incur additional costs.
                    \end{itemize} \\ \hline
    \end{tabular}
    \caption{Advantages and disadvantages of MySQL}
\end{table}

\subsubsection{MongoDB}

MongoDB is designed for applications that handle both structured and unstructured data, and it is offered in both free and paid versions. The highly adaptable database engine establishes connections between databases and applications using MongoDB database drivers, providing a wide array of options that are compatible with multiple programming languages. Although MongoDB may encounter performance challenges when heavily utilized for relational data models, it demonstrates exceptional capabilities in managing changeable, non-relational data. Successive iterations consistently incorporate functionalities to cater to various applications, bolster operational robustness, and prioritize the protection of data.

\begin{table}[H]
    \centering
    \begin{tabular}{| c | p{0.4\textwidth} | p{0.4\textwidth} |}
        \hline
        \multicolumn{1}{|c|}{}
        & \multicolumn{1}{c|}{Advantage}
        & \multicolumn{1}{c|}{Disadvantage} \\ \hline
        \multirow{7}{*}{MongoDB}     
                &   \begin{itemize}[leftmargin=*,topsep=0pt,partopsep=0pt,parsep=0pt]
                        \item MongoDB is fast and easy to use.
                        \item The engine supports JSON and other NoSQL document formats.
                        \item Capable of effectively storing and retrieving data of any arrangement without rigid schema prerequisites.
                        \item  Allows the modification of schema without downtime, facilitating flexibility.
                    \end{itemize}
                &   \begin{itemize}[leftmargin=*,topsep=0pt,partopsep=0pt,parsep=0pt]
                        \item Tools translating SQL to MongoDB queries add an extra step in using the engine.
                        \item The default settings may lack sufficient security measures, necessitating further tweaking to provide optimal security.
                    \end{itemize} \\ \hline
    \end{tabular}
    \caption{Advantages and disadvantages of MongoDB}
\end{table}

\subsubsection{Oracle}

Oracle is a prominent and long-lasting database management solution that has been in use since the late 1970s. The different versions of the software are designed to meet the specific requirements of different organizations. The current long-term release focuses on providing comprehensive support and ensuring reliability. The most recent update incorporates cutting-edge functionalities like as autonomic management, AutoML, and improved multi-model compatibility, hence strengthening its attractiveness for future use.

\begin{table}[H]
    \centering
    \begin{tabular}{| c | p{0.4\textwidth} | p{0.4\textwidth} |}
        \hline
        \multicolumn{1}{|c|}{}
        & \multicolumn{1}{c|}{Advantage}
        & \multicolumn{1}{c|}{Disadvantage} \\ \hline
        \multirow{7}{*}{Oracle}     
                &   \begin{itemize}[leftmargin=*,topsep=0pt,partopsep=0pt,parsep=0pt]
                        \item Oracle consistently deliver cutting-edge innovations.
                        \item Oracle's tools are highly robust, offering a wide range of capabilities to meet diverse requirements.
                        \item Supports various data models, including semistructured (JSON, XML), spatial, RDF, and structured data (SQL).
                        \item Provides multiple access patterns based on the data model.
                    \end{itemize}
                &   \begin{itemize}[leftmargin=*,topsep=0pt,partopsep=0pt,parsep=0pt]
                        \item The cost of Oracle can be prohibitive, particularly for smaller organizations.
                        \item The installation process may need a substantial amount of resources, which could result in the need for hardware upgrades.
                    \end{itemize} \\ \hline
    \end{tabular}
    \caption{Advantages and disadvantages of Oracle}
\end{table}

\subsubsection{PostgreSQL}

To summarize, PostgreSQL is a popular and freely available database system that is extensively utilized for web-based databases. Being one of the pioneering database management systems, it supports both organized and unstructured data and is compatible with multiple platforms, including Linux. The most recent update incorporates improvements in compression choices, backing for organized server log output in JSON format, and general benefits in performance.

\begin{table}[H]
    \centering
    \begin{tabular}{| c | p{0.4\textwidth} | p{0.4\textwidth} |}
        \hline
        \multicolumn{1}{|c|}{}
        & \multicolumn{1}{c|}{Advantage}
        & \multicolumn{1}{c|}{Disadvantage} \\ \hline
        \multirow{10}{*}{PostgreSQL}     
                &   \begin{itemize}[leftmargin=*,topsep=0pt,partopsep=0pt,parsep=0pt]
                        \item PostgreSQL has excellent scalability, enabling efficient management of terabytes of data.
                        \item The system is capable of handling JSON data format, which allows for versatile data administration.
                        \item Offers a range of predefined functions for diverse database operations.
                        \item Multiple interfaces are available for user interaction.
                        \item Functions as a multi-model database, supporting Spatial Data, Key-Value, Structured Data (SQL), and Semi-Structured Data (JSON, XML).
                    \end{itemize}
                &   \begin{itemize}[leftmargin=*,topsep=0pt,partopsep=0pt,parsep=0pt]
                        \item The documentation may be incomplete or challenging to navigate
                        \item Speed may be affected during extensive bulk operations or read queries involving large datasets.
                    \end{itemize} \\ \hline
    \end{tabular}
    \caption{Advantages and disadvantages of PostgreSQL}
\end{table}

\subsubsection{Conclusion}

After analyzing the advantages and disadvantages mentioned, it is clear that each of the four databases has unique strengths and limitations that are suited to certain project needs. When it comes to situations where the most important factors are scalability, flexibility, and excellent performance, MongoDB is the ideal choice. MongoDB is widely recognized for its remarkable capacity to handle large amounts of data and heavy website traffic. MongoDB's document-oriented architecture is specifically tailored for unstructured data and complex data structures. This design provides efficient access and retrieval of data, resulting in improved overall performance.