\section{THEORETICAL BASIS}

\subsection{Model}

\subsubsection{Model-View-Controller (MVC)}

MVC is an architectural design pattern for building applications that separates the application into three main logical components: the model, the view, and the controller. In the MVC model:

\begin{itemize}
    \item \textbf{Model}: Represents the data and business logic of the application. It is the "unchanging essence" of the application and should be stable and long-lived.

    \item \textbf{View}: Represents the user interface of the application. There can be multiple views for different devices, such as a web interface, mobile app, or command line interface.

    \item \textbf{Controller}: Handles user input and updates the model and view accordingly. It acts as an intermediary between the model and the view, processing business logic and managing data flow.
\end{itemize}

\begin{figure}[H]
    \centering
    \includegraphics[width=\textwidth]{assets/images/Research/System/MVC.png}
    \caption{MVC architecture}
    \label{fig:mvc}
\end{figure}

\subsubsection{Model-View-ViewModel (MVVM)}

MVVM architecture offers two-way data binding between view and view-model. The view-model makes use of observer pattern to make changes in the view-model. The components of MVVM are:

\begin{itemize}
    \item \textbf{Model}: Holds the data and business logic. The model is independant of UI or user interaction.
    \item \textbf{View}: Represents how the user interacts with the application.
    \item \textbf{View-model}: Acts as a bridge between the view and the model. It provides data from the model to the view and handles user interaction with the view.
\end{itemize}

\begin{figure}[H]
    \includegraphics[width=\textwidth]{assets/images/Research/System/MVVM.png}
    \caption{MVVM architecture}
    \label{fig:mvvm}
\end{figure}

\subsubsection{Conclusion}


\begin{table}[H]
    \centering
    \begin{tabular}{| c | p{0.4\textwidth} | p{0.4\textwidth} |}
        \hline
        \multicolumn{1}{|c|}{}
        & \multicolumn{1}{c|}{Advantage}
        & \multicolumn{1}{c|}{Disadvantage} \\ \hline
        \multirow{4}{*}{MVC}     
                &   \begin{itemize}[leftmargin=*,topsep=0pt,partopsep=0pt,parsep=0pt]
                        \item The development of the various components can be performed parallelly.
                        \item It avoids complexity by dividing an application into separate (MVC) units
                        \item It only uses a front controller pattern that processes web application requests using a single controller.
                        \item It provides a clean separation of concerns.
                    \end{itemize}
                &   \begin{itemize}[leftmargin=*,topsep=0pt,partopsep=0pt,parsep=0pt]
                        \item Business logic is mixed with Ul
                        \item Hard to reuse and implement tests
                        \item There is a need for multiple programmers to conduct parallel programming.
                    \end{itemize} \\ \hline
        \multirow{4}{*}{MVVC}
                &   \begin{itemize}[leftmargin=*,topsep=0pt,partopsep=0pt,parsep=0pt]
                        \item Business logic is decoupled from Ul.
                        \item Easy to reuse components.
                        \item You can write unit test cases for both the viewmodel and Model layer without the need to reference the View.
                    \end{itemize}
                &   \begin{itemize}[leftmargin=*,topsep=0pt,partopsep=0pt,parsep=0pt]
                        \item Maintenance of lots of codes in controller
                        \item Some people think that for simple UIs of MVVM architecture can be overkill
                    \end{itemize} \\ \hline
    \end{tabular}
\end{table}

Base on the comparison above, MVVM is more suitable for this project because it is easier to reuse components and write unit test cases.

\subsection{Frontend}

\subsubsection{Objective C}

Objective-C currently ranks 22nd in popularity according to the TIOBE Index. While its use is declining, it remains significant in the Apple ecosystem, with around 60,000 active repositories. Introduced in the 1980s by Brad Cox and Tom Love, Objective-C is an object-oriented programming language primarily used for Apple platforms. It was licensed by NeXT Computer Inc., a company founded by Steve Jobs. Noteworthy features include classes, inheritance, dynamic binding, and a Smalltalk-like syntax for messaging objects. Despite declining usage, Objective-C remains one of the more popular programming languages in the market

Some of the great features of this language are:

\begin{itemize}
    \item \textbf{Object-Oriented}: Supports classes, objects, and object-oriented concepts.
    \item \textbf{Dynamic Binding}: Resolves method calls at runtime for flexibility.
    \item \textbf{Message Passing}: Utilizes message passing for a natural coding style.
    \item \textbf{Smalltalk-style Syntax}: Simple and expressive syntax like Smalltalk for clear code.
    \item \textbf{Cross-Platform}: Works on macOS, iOS, and other Apple platforms, plus Windows and Linux.
    \item \textbf{Interoperability with C}: Integrates seamlessly with C libraries for collaboration.
\end{itemize}

\subsubsection{Swift}

Swift is a powerful and modern programming language developed by Apple for building applications across its ecosystem, including iOS, macOS, watchOS, and tvOS. Introduced in 2014, Swift was designed to be efficient, expressive, and easy to learn, addressing the shortcomings of its predecessor, Objective-C. With a syntax that is both concise and readable, Swift empowers developers to create robust and high-performance applications. Known for its safety features, dynamic capabilities, and versatility, Swift has quickly become the language of choice for many developers seeking to craft innovative and seamless experiences within the Apple ecosystem. In this introduction, we'll explore the key features and characteristics that make Swift a standout language in the world of software development.

\subsubsection{SwiftUI}

In summary, SwiftUI, developed by Apple, has become increasingly popular among iOS app developers for valid reasons. Introduced five years after Swift alongside Swift 5 and Xcode 11, it serves as a UI toolkit designed for creating software across various platforms, including iOS, macOS, watchOS, and tvOS. SwiftUI offers a declarative approach to UI design, enabling developers to describe layout and behavior using a simple and intuitive syntax. This results in more efficient and faster development of complex UIs, thanks to concise and easily readable code compared to traditional imperative approaches.

In summary, SwiftUI stands out with these key features:

\begin{itemize}
    \item \textbf{Declarative Syntax}: Describes UI appearance easily, improving code readability.
    \item \textbf{Automatic Layout}: Handles UI layout automatically, reducing manual effort.
    \item \textbf{Dynamic UI}: Enables the creation of dynamic and interactive interfaces.
    \item \textbf{Cross-platform Support}: Works across all Apple devices with a single codebase.
    \item \textbf{Real-time Preview}: Provides live previews of UI changes during development.
    \item \textbf{Pre-built Components}: Offers ready-to-use UI components for quick and efficient development.
\end{itemize}

\subsubsection{Conclusion}

\begin{table}[H]
    \centering
    \begin{tabular}{| >{\centering\arraybackslash}m{2cm} | >{\centering\arraybackslash}m{6cm} | >{\centering\arraybackslash}m{6cm} |}
        \hline
        \textbf{Feature} & \textbf{Objective C} & \textbf{Swift} \\ \hline
        Age & Developed in the early 1980s & Introduced in 2014 \\ \hline
        Syntax & Uses C-based syntax with Smalltalk-style messaging & Uses a modern and concise syntax \\ \hline
        Performance & Slower than Swift due to overhead and lack of optimization & Faster than Objective-C due to optimization features \\ \hline
        Memory Managment & Prone to memory leaks & No memory leaks due to being type-safe and memory-safe \\ \hline
        Stability & Stable & Unstable as it is still growing \\ \hline
    \end{tabular}
\end{table}

Base on the comparison above, Swift is more suitable for this project because it is superior in multiple aspects when comparing to Objective C.

\subsection{Backend}

\subsubsection{Java}



\subsubsection{Python}



\subsection{Database}

\subsubsection{MySQL}



\subsubsection{MongoDB}



\subsubsection{Oracle}



\subsubsection{PostgreSQL}

