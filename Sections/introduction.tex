%
% This is the TeX file for the introduction section
%
\section{INTRODUCTION}

\subsection{Motivation}

Ho Chi Minh City is a thriving metropolis known for its vibrancy and economic importance, but it also has a persistent problem with traffic congestion. An astounding estimated cost of 6 billion USD per year \cite{congestion-damage} highlights the crippling effects of traffic congestion on Ho Chi Minh City, indicating not only the financial toll on businesses but also the pressing need to put into practice practical solutions for sustainable urban development. Within this context, the project's driving force is the necessity to improve the current mobile application aimed at facilitating better traffic conditions in Ho Chi Minh City. \cite{utraffic-mobile}

\begin{figure}[H]
    \includegraphics[width=\textwidth]{assets/images/Trivia/Ket_xe.jpg}
    \caption{Traffic congestion at the Hang Xanh intersection at the end of 2022.}
    \label{fig:ket_xe}
\end{figure}

The current measures taken by the government to alleviate traffic congestion in Ho Chi Minh City have mostly failed, despite their concerted efforts. The rapid urbanization and increasing vehicular density have posed challenges for infrastructure projects and traffic management initiatives to keep up with. Moreover, the current deficiencies in the public transportation system have hindered its capacity to function as a viable alternative for a significant portion of the population. The insufficient effectiveness of these measures underscores the intricacy of the challenge, requiring a thorough reassessment and inventive strategy to attain significant and enduring outcomes.

\subsection{Available solutions}

In response to the pressing issue of traffic congestion in Ho Chi Minh City, various measures have been implemented to alleviate the strain on the urban transportation system. One notable initiative involves ongoing investments in infrastructure development, with the construction of new roads and the expansion of existing thoroughfares aimed at accommodating the city's growing population and vehicular density. Additionally, the government has introduced and expanded public transportation services, such as buses and the metro system, as a means to encourage citizens to opt for more sustainable modes of commuting.

A different strategy for addressing traffic congestion involves the creation of specialized programs for the purpose of continuously monitoring the situation in real-time. These applications utilize technology to deliver real-time updates on traffic conditions, acknowledging the demand for immediate information. Through the collection and analysis of data from diverse sources, including GPS-enabled devices and surveillance cameras, these applications enable users to make well-informed decisions regarding their travels. An example is the TTGT HCMC application created by the Department of Transportation in Ho Chi Minh City. This application functions as a vital instrument, providing users with instant access to real-time updates on the status of routes and visual information through a network of 685 strategically positioned cameras throughout the city. These innovative technologies enhance the ability to effectively manage traffic congestion in urban environments by employing a dynamic and adaptable approach.

\subsection{Problem statement}

UTraffic \cite{utraffic-introduction} is a city traffic prediction system that relies on data gathered from the local population developed by Professor Tran Minh Quang, alongside his associates and students at Ho Chi Minh University of Technology. UTraffic's products aim to mitigate traffic congestion and enhance safety and convenience for users. Using big data analytics and machine learning, the solutions estimate and forecast traffic conditions accurately and quickly, warn users of traffic jams, support efficient routing that takes traffic conditions into account, and provide statistical information and forecasts to help managers make decisions. \\
However, despite UTraffic's commendable objectives and the overarching focus on the mobile application and backend technologies, several challenges have surfaced in the system's current implementation that require careful consideration and resolution:

\begin{itemize}
    \item Observations from the previous authors highlight the necessity for refactoring the code base of both the server and mobile app to enhance reusability and readability. Furthermore, the documentation requires centralization to simplify the installation and configuration process for new developers.\cite{refactor_needed}
    
    \item In addition to impeding thorough data collecting and insights into traffic patterns in Ho Chi Minh City, the lack of an iOS version restricts our ability to interact with the sizable iPhone user base.
    
    \item In addition to relying on crowd-sourced data for efficient routing, our existing approaches to traffic congestion are not sufficiently diverse.
\end{itemize}


\subsection{Scope and objectives}

The UTraffic app and backend developed by previous groups serve as the basis for our efforts to enhance and innovate Ho Chi Minh City's traffic circumstances. The objective of the project is to expand the number of users by creating an iOS version of the current Android application. The objective of this iOS expansion is to enhance the accessibility of the UTraffic application, thereby encouraging a broader range of users. Objectively, the introduction of supplementary bus-related services is intended to encourage the greater use of public transportation. These services will provide users with up-to-date information about bus routes, schedules, and other information, improving the attractiveness of sustainable commuting choices. Moreover, a crucial component of the project entails the reorganization of the code base, with a focus on enhancing the internal code structure to achieve better clarity, readability, and reusability. This aims to streamline development and enhance long-term maintenance efficiency. At the same time, with the emergence of research groups with similar interest in the traffic field, making the UTraffic mobile app in particular and the whole UTraffic system in general a place to showcase and deliver those new features will be one of our focus points. The project aims to make a substantial contribution to the improvement of urban mobility in Ho Chi Minh City through these collaborative efforts.